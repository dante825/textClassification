%!TEX root=thesis.tex
%This is the draft abstract. More soon, I promise!
Pengklasifikasian teks adalah salah satu tugas utama dalam pemprosesan bahasa semulajadi (NLP). Klasifikasi teks mengklasifikasi teks ke dalam beberapa kategori. Kaedah perwakilan dokumen yang digunakan untuk mengekstrak ciri dari teks biasanya menghasilkan matriks yang besar dan jarang. Dimensi matriks yang besar adalah halangan kepada ketepatan model klasifikasi. Oleh itu, algoritma pengurangan dimensi akan digunakan untuk mengurangkan dimensi matriks yang besar dan jarang ini. Kajian ini menganalisis kesan pengurangan dimensi pada matriks dari algoritma perwakilan dokumen yang berlainan dan seterusnya prestasi model klasifikasi yang dilatih dari matriks besar dan jarang asal dan matriks yang terhasil daripada algoritma pengurangan dimensi yang berbeza. Prestasi model klasifikasi dinilai untuk mengenal pasti algoritma perwakilan dimensi, algoritma pengurangan dimensi dan model klasifikasi yang optimum.
