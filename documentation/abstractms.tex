%!TEX root=thesis.tex
%This is the draft abstract. More soon, I promise!
Pengklasifikasian teks adalah salah satu tugas utama dalam pemprosesan bahasa semulajadi. Klasifikasi teks memberikan teks ke dalam kategori yang telah ditentukan sebelumnya. Kaedah perwakilan dokumen yang digunakan untuk mengekstrak ciri dari teks biasanya menghasilkan matriks yang besar dan jarang. Keamatan dimensi matriks akan menjadi penghalang kepada ketepatan model klasifikasi. Oleh itu, algoritma pengurangan dimensi digunakan untuk mengurangkan dimensi matriks besar dan jarang ini. Kajian ini menganalisis kesan pengurangan dimensi pada matriks dari algoritma perwakilan dokumen yang berlainan dan seterusnya prestasi model klasifikasi yang dilatih dari matriks besar dan jarang asal dan matriks yang terhasil daripada algoritma pengurangan dimensi yang berbeza. Prestasi model klasifikasi dinilai untuk mengenal pasti algoritma perwakilan dimensi yang dioptimumkan, algoritma pengurangan dimensi dan model klasifikasi.
